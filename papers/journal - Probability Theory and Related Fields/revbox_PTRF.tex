\RequirePackage{fix-cm}
\RequirePackage{amsmath}

\documentclass[smallextended]{svjour3} 

\smartqed  % flush right qed marks, e.g. at end of proof
%
\usepackage{graphicx}
\usepackage{epstopdf}
\usepackage{amssymb,amsmath}
%
% \usepackage{mathptmx}      % use Times fonts if available on your TeX system
%
% insert here the call for the packages your document requires
%\usepackage{latexsym}
% etc.
%
% please place your own definitions here and don't use \def but
% \newcommand{}{}
%
% Insert the name of "your journal" with
% \journalname{myjournal}
%
\begin{document}

\title{Revisited Box Counting Technique in Bayesian Sense}
\subtitle{}

%\titlerunning{Revised Box Counting}        % if too long for running head

\author{David Blatsk\'{y}         \and
        Jarom\'{i}r Kukal
}

%\authorrunning{D. Blatský} % if too long for running head

\institute{D. Blatsk\'{y} \and  \at
              Department of Software Engineering, Faculty of Nuclear Sciences and Physical Engineering, Czech Technical University in Prague\\
              Tel.: +420-728-122194\\
              \email{blatsdav@fjfi.cvut.cz}
           \and
           J. Kukal \at
              Department of Software Engineering, Faculty of Nuclear Sciences and Physical Engineering, Czech Technical University in Prague
}

\date{Received: date / Accepted: date}
% The correct dates will be entered by the editor


\maketitle

\begin{abstract}
Fractal patterns appear in a wide variety of sources across nature. The unusual characteristic of fractals is that they entail non-integer dimension. The Box Counting method is one of the often used approach to estimate the fractal dimension of a signal. Thanks to the relationship between entropy and the fractal dimension, it is possible to employ entropy in estimating the fractal dimension. In this paper, we propose to utilize Bayesian estimate of Hartley entropy of a finite sample in fractal dimension estimation. This method was tested on artificial fractals generated by recursive expansion of appropriate matrices.
\keywords{Unbiased estimation \and Hartley entropy \and Shannon entropy \and Box Counting }
% \PACS{PACS code1 \and PACS code2 \and more}
% \subclass{MSC code1 \and MSC code2 \and more}
\end{abstract}


\section {Introduction }
Fractals are objects, which exceed its topological dimensions and its Hausdorff dimension is not integer. Box Counting method is used for estimating fractal dimension in the form
\begin{equation} 
\label{eq:boxcount}
\ln{C(a)} = A - D_{0}\ln{a},
\end{equation}
where $a>0$ is box size and $C(a)$ is number of covering elements. Capacity dimension $D_{0}$ is estimated as slope of the line which is generated by least square method. These estimate is biased especialy for small values of $a$. Box Counting method can be improved by Bayesian estimation of Hartley entropy $H_{0}$, which offers better estimate of capacity dimension $D_{0}$.

\section {Multinomial Distribution and Naive Entropy Estimates}

A multinomial distribution [\ref{bib:Multinomial Dirichlet}] model plays the main role in investigating of point set structures. Let $n \in \mathbb{N}$ be a number of distinguished events. Let $p_{j} > 0$ be a probability of the $j^{\text{th}}$ event for $j = 1,...,n$ satisfying $ \sum_{j=1}^{n} p_{j} =1$. Then the random variable $j$ has a multinomial distribution $\text{Mul}(p_{1},...,p_{n})$. After realization of multinomial distribution sample of size $N \in \mathbb{N}$, we can count the events and obtain $N_{j} \in \mathbb{N}_{0}$ as the number of $j^{\text{th}}$ event occurrences for $j=1,...,n$ satisfying $\sum_{j=1}^{n} N_{j} = N$. Therefore, we define the number of various events in a sample as $K = \sum_{N_{j}>0} 1 \le \text{min}(n,N)$. Revising Hartley [\ref{bib:Renyi}] and Shannon [\ref{bib:Renyi}] entropy definitions
\begin{equation} 
\label{eq:hnula}
H_{0}=\ln{n},
\end{equation} 
\begin{equation} 
\label{eq:hjedna}
H_{1}=-\sum_{j=1}^{n} p_{j}\ln{p_{j}},
\end{equation}   
we can perform a direct but naive estimation of them as
\begin{equation} 
\label{eq:hnulaap}
\hat{H}_{0,\mbox{\scriptsize{naive}}}=\ln{K},
\end{equation} 
\begin{equation} 
\label{eq:hjednaap}
\hat{H}_{1,\mbox{\scriptsize{naive}}}=-\sum_{N_{j}>0} \frac{N_{j}}{N}\ln{\frac{N_{j}}{N}}.
\end{equation}   
The main disadvantage of the naive estimates is their biasness. The random variable $K \in \{ 1,...,n \} $ is capped by $n$, which causes $\text{E}\hat{H}_{0,\mbox{\scriptsize{naive}}} = \text{E}\ln{K} < \text{E}\ln{n} = \ln{n} = H_{0}$. Hence, the naive estimate of Hartley entropy $\hat{H}_{0,\mbox{\scriptsize{naive}}}$ is negatively biased. On the other hand, the traditional Box Counting Technique is based on this estimate. There we plot the logarithm of the covering element number $C(a) \in \mathbb{N}$ against the logarithm of the covering element size $a > 0$ and then estimate their dependency in the linear form $\ln{C(a)} = A_{0} - \hat{D}_{0,\mbox{\scriptsize{naive}}}\ln{a}$. Recognizing equivalence $C(a) = K$ leads to $\ln{C(a)} = \ln{K} = \hat{H}_{0,\mbox{\scriptsize{naive}}}$ and then $\hat{H}_{0,\mbox{\scriptsize{naive}}} = A_{0} - \hat{D}_{0,\mbox{\scriptsize{naive}}}\ln{a}$. Defining $\hat{D}_{0,\mbox{\scriptsize{naive}}}$ as an estimate of capacity dimension and recognizing the occurrence of $\hat{H}_{0,\mbox{\scriptsize{naive}}}$ in the Box Counting procedure [\ref{bib:Box counting}], we are not surprised to be victims of the bias of Hartley entropy estimate.\\ 
\\*
A similar situation is the case of Shannon entropy estimation. There are several approaches how to decrease the bias of $\hat{H}_{1,\mbox{\scriptsize{naive}}}$ to be closer to a theoretical value of Shannon entropy $H_{1}$. Miller [\ref{bib:Harris}] modified the naive estimate $\hat{H}_{1,\mbox{\scriptsize{naive}}}$ using a first-order Taylor expansion resulting in
\begin{equation}
\label{eq:miller}
\hat{H}_{1,\mbox{\scriptsize{M}}}=\hat{H}_{1,\mbox{\scriptsize{naive}}} + \frac{K-1}{2N}.
\end{equation}
Lately, Harris [\ref{bib:Harris}] improved the formula to
\begin{equation}
\label{eq:harris1h}
\hat{H}_{1,\mbox{\scriptsize{H}}}=\hat{H}_{1,\mbox{\scriptsize{naive}}} + \frac{K-1}{2N} - \frac{1}{12N^2} \left( 1 - \sum_{p_{j}>0}\frac{1}{p_{j}} \right).
\end{equation}
Finally, we can estimate the capacity and information dimensions according to relation
\begin{equation} 
\label{eq:hjednaest}
\hat{H}_{d}=A_{d} - \hat{D}_{d} \ln{a},
\end{equation} 
where $\hat{H}_{d}$ is any estimate of $H_{d}$. Therefore, we can also estimate Hausdorff dimension $D_{\mbox{\scriptsize{H}}}$ using inequalities $D_{1} \le D_{\mbox{\scriptsize{H}}} \le D_{0}$ under the assumption that $\hat{D}_{1} \le D_{\mbox{\scriptsize{H}}} \le \hat{D}_{0}$ for any ``good'' estimates $\hat{D}_{0}$, $\hat{D}_{1}$ of capacity and information dimensions, respectively. The next section is oriented to Bayesian estimation of $H_{0}$ and $H_{1}$, which are essential for evaluating $\hat{D}_{0}$ and $\hat{D}_{1}$.

\section {Bayesian Estimation of Hartley Entropy}
We suppose Dirichlet distribution [\ref{bib:Multinomial Dirichlet}] of a random vector $\mathbf{p} = (p_{1},...,p_{n})$ satisfying  $p_{j} \ge 0$, $\sum_{j=1}^{n} p_{j} = 1$, with $\alpha_j = \alpha^{*} > 0$. Using properties of multinomial and its conjugate distribution --- the Dirichlet distribution, we can calculate probability estimate $\mathrm{\hat{p}}(K|n,N)$ of the random variable $K \in \mathbb{N}$ for $K \le \min(n,N)$ as 
\begin{equation} 
\label{eq:pknn}
\begin{split}
\mathrm{\hat{p}}(K \: | \: n,N) & = \text{prob}\left(\sum_{N_{j} > 0}{1}=K \: \middle| \: n,\sum_{j=1}^{n}{N_{j}}=N\right) \\ 
& = \binom{n}{K} \frac{\Gamma({N+1}) \Gamma(n\alpha^{*})}{\Gamma(N+n\alpha^{*})} \sum_{\vec{N} \in \mathbb{D}_{K,N}} \prod_{j=1}^{K} \frac{ \Gamma(N_{j} + \alpha^{*})}{ \Gamma(N_j+1) \Gamma(\alpha^{*})}.
\end{split}
\end{equation}
Derivation of (\ref{eq:pknn}) is included in the Appendix \ref{subsec:app1}. When $N \ge K+2$, we can calculate 
\begin{equation} 
\label{eq:skn}
S_{K,N} = \sum_{n=K}^{\infty}{\mathrm{\hat{p}}\left(K \: \middle| \: n,N\right)}.
\end{equation}
When the number of events is constrained as $n \leq n_{\text{max}} $, we apply an alternative formula
\begin{equation} 
\label{eq:sknalt}
S_{K,N}^{*} = \sum_{n=K}^{n_{\text{max}}}{\mathrm{\hat{p}}\left(K \: \middle| \: n,N\right)}.
\end{equation}
Convergence of the infinite series (\ref{eq:skn}) is proved in the Appendix \ref{subsec:conv}. Having a knowledge of $K,N$ where $N \ge K+2$, we can calculate a Bayesian density
\begin{equation} 
\label{eq:pnkn}
\mathrm{\hat{p}}\left(n \: \middle| \: K,N \right) = \frac{\mathrm{\hat{p}}\left(K \: \middle| \: n,N\right)}{S_{K,N}}, n \ge K
\end{equation}
Thereafter, Bayesian estimate of Hartley entropy comes out as
\begin{equation} 
\label{eq:hnbayes}
\begin{split}
\hat{H}_{0,\mbox{\scriptsize{Bayes}}} & = \text{E}H_{0}  = \sum_{n=K}^{\infty}{\mathrm{\hat{p}}\left(n \: \middle| \: K,N\right)\ln{n}} = \sum_{n=K}^{\infty} \frac{\mathrm{\hat{p}}\left(K \: \middle| \: n,N\right)\ln{n}}{S_{K,N}}  \\
 & =  \frac{ \sum_{n=K}^{\infty} \mathrm{\hat{p}}\left(K \: \middle| \: n,N\right)\ln{n}}{\sum_{n=K}^{\infty} \mathrm{\hat{p}}\left(K \: \middle| \: n,N\right)} > \ln{K},
\end{split}
\end{equation}
which is a convergent sum as well. We gain an equivalent formula by substituting $n=K+j$ \begin{equation} 
\label{eq:hnbayesb}
\hat{H}_{0,\mbox{\scriptsize{Bayes}}} = \frac{\sum_{j=0}^{\infty}{b_{j}\ln{(K+j)}}}{\sum_{j=0}^{\infty}{b_j}},
\end{equation}
where
\begin{equation}
b_{j}= \binom{K+j}{j} \frac{\mathrm{B}\left( (K+j)\alpha^{*}, N \right)}{\mathrm{B}\left( K\alpha^{*}, N \right)}.
\end{equation} 
Convergence of the sums in (\ref{eq:hnbayes}) is proved in Appendix \ref{subsec:conv}. Particular coefficients $b_{j}$ can also be generated recursively
\begin{equation}
\label{eq:breform}
\begin{split}
& b_{0} = 1 \\
& b_{j} = \frac{K+j}{j} \frac{\Gamma{((K+j)\alpha^{*})}}{\Gamma{((K+j)\alpha^{*}-\alpha^{*})}}\frac{\Gamma{(N+(K+j)\alpha^{*} -\alpha^{*})}}{\Gamma({N+(K+j)\alpha^{*})}} b_{j-1} \\
& b_{j} = b_{j-1} \frac{K+j}{j}\prod_{u=0}^{N-1} \left(1 - \frac{\alpha^{*}}{(K+j)\alpha^{*}+u} \right).
\end{split}
\end{equation}
%Asymptotic properties of Bayesian estimate can be investigated for $N \rightarrow +\infty$ via limits
%\begin{equation} 
%\label{eq:lim}
%\begin{split}
%\lim_{N \rightarrow +\infty} & {\hat{H}_{0,\mbox{\scriptsize{Bayes}}}} = \ln{K}, \\
%\lim_{N \rightarrow +\infty} & {(\hat{H}_{0,\mbox{\scriptsize{Bayes}}}-\ln{K})N} = K(K+1)\ln(1+1/K), \\
%\lim_{N \rightarrow +\infty} & {\left(\hat{H}_{0,\mbox{\scriptsize{Bayes}}}-\ln{K}-\frac{K(K+1)\ln(1+1/K)}{N}\right)N^2} = \\
%& \frac{1}{2}\left(K(K+2)(K+1)\left(\ln(K+2)-\ln(K)-2K\ln(K+1)+K\ln(K+2)+K\ln(K)\right)\right),
%\end{split}
%\end{equation}
%Therefore
%\begin{equation} 
%\label{eq:hroz}
%\begin{split}
%\hat{H}_{0,\mbox{\scriptsize{Bayes}}} \approx & \ln{K} + \frac{K(K+1)\ln(1+1/K)}{N} + \\ 
%& \frac{\left(K(K+2)(K+1)\left(\ln(K+2)-\ln(K)-2K\ln(K+1)+K\ln(K+2)+K\ln(K)\right)\right)}{2N^2}
%\end{split}
%\end{equation}
%When $K$ is also large, we can roughly approximate Hartley entropy as
%\begin{equation} 
%\label{eq:hartapp}
%\hat{H}_{0,\mbox{\scriptsize{Bayes}}} \approx \ln{K} + \frac{K+1}{N}
%\end{equation}
%which is very similar to Miller correction (\ref{eq:miller}) in the case of Shannon entropy estimation. Meanwhile formula (\ref{eq:hnbayes}) represents Bayesian estimate of $H_{0}$, formulas (\ref{eq:hnulaap}), (\ref{eq:hartapp}), and (\ref{eq:hroz}) are approximations of zero, first, and second order. \\
%\\*
%Formula (\ref{eq:hnbayesb}) can be also expanded to the form
%\begin{equation} 
%\label{eq:hnbayesbexp}
%\hat{H}_{0,\mbox{\scriptsize{Bayes}}} = \ln{K} + \sum_{j=1}^{\infty} \frac{1}{j} \varphi(K)\left(\frac{K}{N}\right)^{j}
%\end{equation}
%where $\varphi(K)>1$ for all $K \in \mathbb{N}$ and $\lim_{K \rightarrow \infty} \varphi(K) = 1$. Therefore, we obtain lower estimate
%\begin{equation}
%\label{eq:hnbayesbexplow}
%\hat{H}_{0,\mbox{\scriptsize{Bayes}}} > \ln{K} + \sum_{j=1}^{\infty} \frac{1}{j} \left(\frac{K}{N}\right)^{j} = \ln{K} - \ln(1 - \frac{K}{N}) = H_{0,\mbox{\scriptsize{low}}}
%\end{equation}
%which exists for $K < N$.

\section {Bayesian Estimation of Shannon Entropy}
In the case when the number of events $n$ is known, we can perform Bayesian estimation of Shannon entropy as
\begin{equation} 
\label{eq:hjednan}
\hat{H}_{1,\mbox{\scriptsize{n}}} = \text{E}H_{1}(K=n) = -\sum_{j=1}^{n} \left( \frac{N_{j}+1}{N+n} \left( \psi(N_{j}+2) - \psi(N+n+1) \right) \right)
\end{equation}
where $\psi$ is digamma function. But when the number of events $n$ is unkonown, we can use $K$ as lower estimate of $n$ and perform final Bayesian estimation as
\begin{equation} 
\label{eq:hjednab}
H_{1,\mbox{\scriptsize{Bayes}}} = \sum_{n=K}^{\infty}{\text{p}\left(n \: \middle| \:K,N \right)H_{1,\mbox{\scriptsize{n}}}}
\end{equation}
which is also convergent sum for $N \ge K+2$. \\
\\*
Substituing $n=K+j$ we obtain adequating formula.
\begin{equation} 
\label{eq:hjbb}
\hat{H}_{1,\mbox{\scriptsize{Bayes}}} = \frac{\sum_{j=0}^{\infty}b_{j}H_{1,K+j}}{\sum_{j=0}^{\infty}b_{j}}
\end{equation}
Asymptotic expansion of (\ref{eq:hjbb}) unfortunately depends on individual frequences $N_{j}$.

\section {Revisited Box Counting Method}
Let $\mathbb{F} \subset \mathbb{R}^{m}$ be a set of $N$ points placed into $m$-dimensional rectangular grid of element size $a > 0$. Let $\hat{H}_{0,\mbox{\scriptsize{Bayes}}}$ be an unbiased estimate of Hartley entropy $H_{0}$. Fitting the linear model
\begin{equation} 
\label{eq:hlinmod}
\hat{H}_{0,\mbox{\scriptsize{Bayes}}} = A - \hat{D}_{0}\ln{a}
\end{equation}
via the method of least squares is called Revisited Box Counting.\\
\\*
Revisited Box Counting can be modified by %:
%\begin{itemize}
%\item 
using $\hat{H}_{1,\mbox{\scriptsize{Bayes}}}$ instead of $\hat{H}_{0,\mbox{\scriptsize{Bayes}}}$ comes to estimation of information dimension [\ref{bib:Information dimension}] according to
\begin{equation} 
\label{eq:inform}
\hat{H}_{1,\mbox{\scriptsize{Bayes}}} = A - \hat{D}_{1}\ln{a}.
\end{equation}
%\item using non-trivial approximations of $\hat{H}_{0,\mbox{\scriptsize{Bayes}}}$, namely: $\hat{H}_{0,1}, \hat{H}_{0,2}, \hat{H}_{0,\mbox{\scriptsize{low}}}$ instead of $\hat{H}_{0,\mbox{\scriptsize{Bayes}}}$
%\end{itemize}
%Remark: \\
%Using of $\hat{H}_{0,0} \equiv \hat{H}_{0,\mbox{\scriptsize{naive}}}$ instead of $\hat{H}_{0,\mbox{\scriptsize{Bayes}}}$ comes back to traditional Box Counting.

\section {Experimental Part }

The Revisited Box Counting technique will be tested on models of deterministic self-similar 2D fractal sets. They are generated by recursive expansion of binary matrix $\mathbb{G}_{u,v} \in \{ 0, 1 \}^{v \times v} $, where $u$ a is the number of non-zero elements (units), $v>1$ is a matrix dimension, and $v<u<v^2$. \\
\\*
Recursive expansion of $\mathbb{G}_{u,v}$ generates a binary matrix which represents fractal set $\mathbb{F}_{u,v}$ of a similarity dimension $D_{\text{S}} = D_{\text{H}} = D_{0} = D_{1} = \frac{\log{u}}{\log{v}}$. Depth $h$ of recursion depends on $v$ and should be appropriate to computer memory size.\\
\\*
Four testing sets $\mathbb{F}_{3,2}$, $\mathbb{F}_{4,3}$, $\mathbb{F}_{5,3}$, $\mathbb{F}_{8,3}$ were generated by matrices:
\begin{itemize}
\item 

$\mathbb{G}_{3,2} = \begin{bmatrix}
1 & 1 \\
1 & 0 
\end{bmatrix}$ for $h=11$, $\text{dim}(\mathbb{F}_{3,2}) = \frac{\log{3}}{\log{2}} = 1.5850,$

\item 

$\mathbb{G}_{4,3} = \begin{bmatrix}
0 & 1 & 0 \\
1 & 0 & 1 \\
0 & 1 & 0
\end{bmatrix}$ for $h=7$, $\text{dim}(\mathbb{F}_{4,3}) = \frac{\log{4}}{\log{3}} = 1.2619,$

\item 

$\mathbb{G}_{5,3} = \begin{bmatrix}
0 & 1 & 0 \\
1 & 1 & 1 \\
0 & 1 & 0
\end{bmatrix}$ for $h=7$, $\text{dim}(\mathbb{F}_{5,3}) = \frac{\log{5}}{\log{3}} = 1.4650,$

\item 

$\mathbb{G}_{8,3} = \begin{bmatrix}
1 & 1 & 1 \\
1 & 0 & 1 \\
1 & 1 & 1
\end{bmatrix}$ for $h=7$, $\text{dim}(\mathbb{F}_{8,3}) = \frac{\log{8}}{\log{3}} = 1.8928.$
\end{itemize}
Sets $\mathbb{G}_{3,2}$ and $\mathbb{G}_{8,3}$ correspond to Sierpinski gasket and carpet respectively. \\
\\*
At first, adequate point sets of given depth $h$ were generated. Then, they were randomly rotated around the origin, and finally they were randomly shifted. Afterwards, a grid of size $a$ was put on the data points and entropy estimates were calculated. Due to physical interpretation of entropy, the estimates were averaged over 10 realizations and mean values of entropy were calculated. \\
\\*
Various estimates of Hartley entropy for the grid of size $a=5,7,10,15,...,150,200$ are depicted in Fig \ref{fig:frac}. Hartley entropy estimates are similar to each other except for under-biased naive estimate. Estimates $H_{0,\mbox{\scriptsize{BAYES}}}$ and $H_{0,\mbox{\scriptsize{low}}}$ are very similar in these four cases. Corresponding estimates of Shannon entropy are depicted in Fig. \ref{fig:fracshan} in the same range. \\
\\*
As seen in Fig. \ref{fig:frac} and \ref{fig:fracshan}, too small grid size $a \leq 20$ comes to underestimation of $\hat{H}_{0,\mbox{\scriptsize{naive}}}, \hat{H}_{1,\mbox{\scriptsize{naive}}}$, but the other estimates are unfortunately overestimated. Therefore, Revisited Box Counting was applied in the range $30 \leq a \leq 100$.\\
\\*
Using the least square method we obtained various estimates of $D_{0}$ and $D_{1}$ and got a chance to compare them with theoretical values of similarity dimension. The results of estimation are collected in Tabs. \ref{tab:est1} - \ref{tab:est4s}, where $\text{E}D$ is point estimate of a given dimension, $s_{D}$ is its standard deviation, and $p_{\text{value}}$ is probability from t-test of hypothesis
\begin{equation} 
\label{eq:hypo}
\text{H}_{0} : \text{E}D = D_{\text{S}}.
\end{equation}

\section {Conclusion}
In this paper we developed the Bayesian estimator $\hat{H}_{0,\mathrm{Bayes}}$ of Hartley entropy for Dirichlet prior. This estimate enables to estimate $\hat{D}_{0}$ with suppressed bias in comparison with naive box-counting estimate. The novel methodology is based on the box-counting estimate $\hat{D}_{0,\mathrm{naive}}$ which helps to specify the Dirichlet prior and finally reestimate the capacity dimension. This procedure is recommended for 2D structures with $1 \leq \hat{D}_{0} \leq 1.6$ and can be easily extended for information dimension $D_1$ estimation and higher dimensions.

\begin{acknowledgements}
The authors acknowledge the funding from the CTU in Prague, \\ Grant SGS14/208/OHK4/3T/14.
\end{acknowledgements}

\begin{thebibliography}{99}
\vskip12pt
\bibitem{bib1}\label{bib1} todo t.\textit{todo}. todo.
\bibitem{Harris}\label{Harris} Harris, B., \textit{The statistical estimation of entropy in the non-parametric case}. MRC Technical Summary Report, 1975 
\bibitem{Wendel}\label{Wendel} Wendel, J., \textit{Note on the gamma function}. Amer. Math. Monthly,
55(1948), 563-564.

\section {Appendix}


\subsection {Derivation of $\hat{\mathrm{p}}(K|n,N)$ for Hartley Entropy}
\label{subsec:app1}
Let $\mathbb{Q}_{n} = \{ \vec{q} \in (\mathbb{R}_{0}^{+})^{n} | \sum_{j=1}^{n}q_{j} = 1 \}$ be a support set of a Dirichlet-distributed random variable $\vec{p} \in \mathbb{Q}_{n}$ with parameters $\alpha_j$, for $j = 1,...,n$. The conditional probability of an~integer $K$ satisfying $1 \le K \le \min(n,N)$ is 
\begin{equation} 
\label{eq:probpkn}
\text{p}(K \: | \: n,N) = \text{prob}\left(\sum_{N_{j}>0} 1 = K \: \middle| \: n, \sum_{j=1}^{n}N_{j} = N\right).
\end{equation}
The vector of $N_{j}$ can be reorganized to begin with positive values:
\begin{equation} 
\label{eq:probbinom}
\text{p}(K \: | \: n,N) = {n \choose K}\text{prob}\left( \forall j=1,...,n : N_{j} > 0 \Leftrightarrow j \le K \: \middle| \: n, \sum_{j=1}^{K}N_{j}=N\right).
\end{equation}
Let $\mathbb{D}_{K,N} = \{ \vec{x} \in \mathbb{N}^K | \sum_{j=1}^{K}x_{j} = N \}$ be the domain of $\vec{N} = (N_{1},...,N_{K}) \in \mathbb{D}_{K,N}$. Using the mean value of a~multinomial distribution over $\mathbb{Q}_{n}$, we obtain an~unbiased estimate of $\text{p}(K \: | \: n,N)$ as
\begin{equation} 
\label{eq:probbinomexp}
\mathrm{\hat{p}}(K \: | \: n,N) = {n \choose K} \text{E}\left(\sum_{\vec{N} \in \mathbb{D}_{K,N}} {N \choose N_{1},...,N_{K}} \prod_{j=1}^{K}p_{j}^{N_{j}} \prod_{j=k+1}^{n}p_{j}^{0} \right) = {n \choose K} \sum_{\vec{N} \in \mathbb{D}_{K,N}} {N \choose N_{1},...,N_{K}} \text{E}\left( \prod_{j=1}^{K}p_{j}^{N_{j}}\right).
\end{equation}
Using the generalized Beta function
\begin{equation} 
\label{eq:betafce}
B(\vec{x}) = \int_{\vec{p} \in \mathbb{Q}_{m}} \prod_{j=1}^{m} p_{j}^{x_{j}-1} \text{d}\vec{p} = \frac{\prod_{j=1}^{m} \Gamma(x_{j})}{\Gamma(\sum_{j=1}^{m}x_{j})},
\end{equation}
we can calculate
\begin{equation} 
\label{eq:expprod}
\text{E}\left( \prod_{j=1}^{K}p_{j}^{N_{j}} \right) = \frac{\int_{\vec{p} \in \mathbb{Q}_{n}} {B(\vec{\alpha})}^{-1} \prod_{j=1}^{K} p_{j}^{N_{j}+\alpha_{j}-1}  \prod_{j=K+1}^{n} p_{j}^{\alpha_{j}-1} \text{d}\vec{p}}{\int_{\vec{p} \in \mathbb{Q}_{n}}  {B(\vec{\alpha})}^{-1}\prod_{j=1}^{n} p_{j}^{\alpha_{j}-1} \text{d}\vec{p}} = \frac{\Gamma(\alpha)}{\Gamma(N+\alpha)} \prod_{j=1}^{K} \frac{\Gamma(N_{j}+\alpha_j)}{\Gamma(\alpha_j)},
\end{equation}
where $\alpha$ is the sum of all $\alpha_j$. Therefore
\begin{equation} 
\label{eq:prob}
\mathrm{\hat{p}}(K \: | \: n,N) = {n \choose K}\sum_{\vec{N} \in \mathbb{D}_{K,N}} \frac{N!}{\prod_{j=1}^{K}N_{j}!}\frac{\Gamma(\alpha)}{\Gamma(N+\alpha)} \frac{\prod_{j=1}^{K} \Gamma({ N_{j} + \alpha_j})}{\prod_{j=1}^{K} \Gamma({\alpha_j})} = {n \choose K} \frac{\Gamma({N+1}) \Gamma(\alpha)}{\Gamma(N+\alpha)} \sum_{\vec{N} \in \mathbb{D}_{K,N}} \prod_{j=1}^{K} \frac{ \Gamma(N_{j} + \alpha_j)}{ \Gamma(N_j+1) \Gamma(\alpha_j)}
\end{equation}
In this particular paper, we assume $\alpha_j = \alpha^{*}, \forall j = 1,...,n$ which results in easier form of the equation \ref{eq:prob}:
\begin{equation} 
\label{eq:probRes}
\mathrm{\hat{p}}(K \: | \: n,N) = {n \choose K} \frac{\Gamma({N+1}) \Gamma(n\alpha^{*})}{\Gamma(N+n\alpha^{*})} \sum_{\vec{N} \in \mathbb{D}_{K,N}} \prod_{j=1}^{K} \frac{ \Gamma(N_{j} + \alpha^{*})}{ \Gamma(N_j+1) \Gamma(\alpha^{*})}
\end{equation}

%\subsection{Matlab library}

\subsubsection{Bayesian estimate $ H_{0,\mbox{\scriptsize{Bayes}}}$ }
\ttfamily
\begin{lstlisting}
function H0=HARTLEYBAYES(Nvec, alpha) 
    N = sum(Nvec); K = sum(Nvec>0);        
    bay=1; bay0=log(K); H0=bay0/bay; b=1;
    for j = 1 : length(Nvec)-K
       H0old=H0;   
       b=b/j*(K+j) * prod(ones(1, N)-alpha./((K+j)*alpha:N-1+(K+j)*alpha));
       bay=bay+b;
       bay0=bay0+b*log(K+j);
       H0=bay0/bay;
       if bay>1e200
           bay0=bay0/bay;
           b=b/bay;
           bay=1;           
       end
       if abs(H0-H0old) < 1e-8
           break
       end
    end
end
\end{lstlisting}

\normalfont
\subsubsection{Bayesian estimate $ H_{1,\mbox{\scriptsize{Bayes}}}$ }
\ttfamily
\begin{lstlisting}
function H1=SHANNONBAYES(Nvec, alpha)
    N=sum(Nvec); K=sum(Nvec>0);
    Nvec=Nvec(Nvec>0);
    bay=1; bay1=SHANNONFIXED(Nvec); H1=bay1/bay; b=1;
    for j=1: length(Nvec)-K
        H1old=H1;
        b=b/j*(K+j) * prod(ones(1, N)-alpha./((K+j)*alpha:N-1+(K+j)*alpha));
        bay=bay+b;
        Nvec=[Nvec,0];
        bay1=bay1+b*SHANNONFIXED(Nvec);
        H1=bay1/bay;
        if bay>1e200;
            bay1=bay1/bay;
            b=b/bay;
            bay=1;
        end
        if abs(H1-H1old)< 1e-8
            break
        end
    end
end
\end{lstlisting}

\normalfont
\subsubsection{Bayesian estimate $ H_{1,\text{n}}$ }
\ttfamily
\begin{lstlisting}
function H1=SHANNONFIXED(Nvec)
	N=sum(Nvec);
	n=length(Nvec);
	H1=((Nvec+1)/(N+n))'*(psi(N+n+1)-psi(Nvec+2));
end
\end{lstlisting}

\normalfont
\ttfamily
\begin{lstlisting}

\end{lstlisting}






\end{document}